\documentclass[11pt,mathserif]{beamer}

\usepackage{amsmath,amssymb,amsfonts}
\usepackage{graphicx}
\newcommand{\tiret}{\rule[0.6ex]{1.3ex}{0.22ex}}

% !! pour le francais
\usepackage[utf8]{inputenc}
\usepackage[english]{babel}
\usepackage{listings}
\usepackage{relsize}
\usepackage{color}
\usepackage{fancyvrb}
\usepackage{emerald}
\usepackage{punk}



%%%%% for smiley
\def\mysmile#1{{\ooalign{\hfil\lower.06ex % a smiley face
 \hbox{$\scriptscriptstyle#1$}\hfil\crcr
 \hfil\lower.7ex\hbox{\"{}}\hfil\crcr
 \mathhexbox20D}}}


\newcommand{\bad}{\bf \textcolor{red}{\mysmile \frown}}
\newcommand{\neutral}{\bf \textcolor{blue}{\mysmile \minus}}
\newcommand{\good}{\bf \textcolor{green}{\mysmile \smile}}

\definecolor{dkgreen}{rgb}{0,0.6,0}
\definecolor{gray}{rgb}{0.5,0.5,0.5}
\definecolor{mauve}{rgb}{0.58,0,0.82}
\lstdefinelanguage{julia}
 {
 morekeywords=\color{blue}]{if,end,else, for, in, begin, type, function, return, Float64, Int64, <:},
 %morekeywords=[\color{yellow}]{->,=}
 sensitive=true,
 morecomment=[l]{\#},%
 morestring=[b]"
 }

\lstset{ %
  language=julia
}

%% ====== my beamer ==
\mode<presentation> {
\usetheme{default}    % sobre
\useinnertheme[shadow]{rounded}  % les numeros
}
\usefonttheme{structurebold}

\begin{document}

%****************************************************************
% Page de presentation 
%**************************************************************
\begin{frame}
\begin{center}
{\Large Introduction to the Julia language }
\end{center}
\vspace{1cm}
\includegraphics[width=0.5\linewidth]{figures/julia.png}
\vspace{1cm}
\begin{center}
{\large Marc Fuentes - SED Bordeaux \\}
\end{center}
\end{frame}

%****************************************************************
% Introduction 
%**************************************************************

\begin{frame}[fragile]
\frametitle{Outline}
\begin{enumerate}[<+->]
\item \julia as a numerical language 
\item some caracteristic about the language
\item examples 
\item performance
\end{enumerate}
\end{frame}
\end{document}
