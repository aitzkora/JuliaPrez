\documentclass[11pt,mathserif]{beamer}

\usepackage{amsmath,amssymb,amsfonts}
\usepackage{graphicx}
\newcommand{\tiret}{\rule[0.6ex]{1.3ex}{0.22ex}}

% !! pour le francais
\usepackage[latin1]{inputenc}
\usepackage[english]{babel}
\usepackage{listings}
\usepackage{relsize}
\usepackage{color}
\usepackage{fancyvrb}
\usepackage{punk}

\newcommand{\fleche}{\alert{$\pmb{\longrightarrow}$}~~}
\newcommand{\scipy}{\textbf{Scipy}{} }
\newcommand{\numpy}{\textbf{Numpy}{} }
%\newcommand{\julia}{{\ECFAugie Julia}{} }
%\newcommand{\matlab}{{\ECFTeenSpirit Matlab}{} }
\newcommand{\julia}{{ Julia}{} }
\newcommand{\matlab}{{ Matlab}{} }
\newcommand{\octave}{\textbf{Octave}{} }
\newcommand{\scilab}{\textbf{Scilab}{} }
\newcommand{\python}{\textbf{Python}{} }

%%%%% for smiley
\def\mysmile#1{{\ooalign{\hfil\lower.06ex % a smiley face
 \hbox{$\scriptscriptstyle#1$}\hfil\crcr
 \hfil\lower.7ex\hbox{\"{}}\hfil\crcr
 \mathhexbox20D}}}

\newcommand{\bad}{\bf \textcolor{red}{\mysmile \frown}}
\newcommand{\neutral}{\bf \textcolor{blue}{\mysmile \minus}}
\newcommand{\good}{\bf \textcolor{green}{\mysmile \smile}}

\definecolor{dkgreen}{rgb}{0,0.6,0}
\definecolor{gray}{rgb}{0.5,0.5,0.5}
\definecolor{mauve}{rgb}{0.58,0,0.82}
\lstdefinelanguage{julia}
 {
 morekeywords=\color{blue}]{if,end,else, for, in, begin, type, function, return, Float64, Int64, <:},
 %morekeywords=[\color{yellow}]{->,=}
 sensitive=true,
 morecomment=[l]{\#},%
 morestring=[b]"
 }

\lstset{ %
  language=julia,
  framerule=0pt,
  basicstyle=\relsize{-2}\ttfamily,  % the size of the fonts that are used for the code
  %backgroundcolor=\color{black!10},  % choose the background color. You must add \usepackage{color}
  showspaces=false,               % show spaces adding particular underscores
  showstringspaces=false,         % underline spaces within strings
  showtabs=false,                 % show tabs within strings adding particular underscores
  %frame=single,                   % adds a frame around the code
  rulecolor=\color{black},        % if not set, the frame-color may be changed on line-breaks within not-black text (e.g. commens (green here))
  breakatwhitespace=false,        % sets if automatic breaks should only happen at whitespace
  keywordstyle=\color{blue},      % keyword style
  commentstyle=\color{dkgreen},   % comment style
  stringstyle=\color{mauve}  
}

%% ====== my beamer ==
\mode<presentation> {
  \usetheme{default}    % sobre
  \useinnertheme[shadow]{rounded}  % les numeros
}
\usefonttheme{structurebold}

\begin{document}

%****************************************************************
% Page de presentation 
%****************************************************************
\begin{frame}
  \begin{center}
    {\Large Introduction to the \julia language }
  \end{center}
  \vspace{1cm}
  \begin{center}
    \includegraphics[width=0.4\linewidth]{figures/julia.png}
  \end{center}
  \vspace{1cm}
  \begin{center}
    {\large Marc Fuentes - SED Bordeaux \\}
  \end{center}
\end{frame}

%****************************************************************
% Plan 
%**************************************************************

\begin{frame}
  \frametitle{Outline}
  \begin{enumerate}[<+->]
    \item motivations
    \item \julia as a numerical language 
    \item types and methods 
    \item about performance
  \end{enumerate}
\end{frame}


%****************************************************************
% Introduction 
%**************************************************************

\begin{frame}
  \frametitle{Why yet another \matlab-like language?}
  \begin{itemize}[<+->]
    \item {\em scripting languages} such as \matlab, \scipy, \octave etc... are
    efficient to prototype algorithms (more than C++/Fortran)
    \item \matlab is not free neither open source. It still remains the {\em lingua franca}
    in numerical algorithms. 
    \item \octave is very slow (but highly compatible with \matlab)
    \item \python along \numpy and \scipy is a beautiful language, but a little bit slow and do not
    support any parallelism as a built-in feature. 
    \item \textbf{Pypy} which is a nice JIT compiler of \python, but does not support neither \numpy neither \scipy
    \item \textbf{R} is well suited for statistics, but suffer from old-syntax troubles
  \end{itemize}
  \pause
  \fleche Why do not try a new language for numerical computation?
\end{frame}


%****************************************************************
% A language for numerical computation 
%**************************************************************

\begin{frame}[fragile]
  \frametitle{A language for numerical computation}
  \begin{itemize}[<+->]
    \item \julia's syntax is very similar to langages as \matlab, \python or \scilab, so
    switching to \julia is fast
    \item do not require {\em vectorized} code to run fast(JIT compiler)
    \item it uses references (also for function arguments)
    \item indices start to 1 and finish to \texttt{end}
    \item use brackets \texttt{[,]} for indexing
    \item it supports broadcasting
    \item support 1D arrays
  \end{itemize}
  \pause
  \fleche let us have a look to some examples
\end{frame}

%****************************************************************
% Functional aspects
%****************************************************************

\begin{frame}[fragile]
  \frametitle{Functional aspects}
  \begin{itemize}[<+->]
    \item support anonymous functions : \texttt{ (x -> x*x)(2) }
    \item support \texttt{map}, \texttt{reduce}, \texttt{filter} functions
    \item functions support variadic arguments (using tuples)
    \item comprehension lists
    \item functions are not supposed to modify their arguments, otherwise
    they follow the \texttt{!} convention like \texttt{sort!}
  \end{itemize}
\end{frame}

%****************************************************************
% Parallelism
%****************************************************************
\begin{frame}[fragile]
  \frametitle{Parallelism}
  \pause
  \begin{itemize}[<+->]
    \item \julia has a built-in support for a distributed memory parallelism
    \item one-sided message passing routines 
    \begin{itemize}
      \item \texttt{remotecall} to launch a computation on a given process 
      \item \texttt{fetch} to retrieve informations 
    \end{itemize}     

    \item  high level routines
    \begin{itemize}
      \item \texttt{@parallel} reduction for lightweight iterations
      \item \texttt{pmap} for heavy iterations
    \end{itemize}     
    \item support for distributed arrays in the standard library 
  \end{itemize}
\end{frame}

%****************************************************************
% external libraries calls
%****************************************************************
\begin{frame}[fragile]
  \frametitle{external libraries calls}
  \pause
  \begin{itemize}[<+->]
    \item sometimes you need to call a C/Fortran code
    \item "no boilerplate" philosophy : do not require { Mexfiles}, Swig or other wrapping system
    \item the code must be in a {\bf shared} library 
    \item the syntax is the following  
  \end{itemize}
  \pause
  \begin{lstlisting}
    ccall(:function, "lib"), return_type, (type_1,...,type_n), args)
  \end{lstlisting}
\end{frame}

%****************************************************************
% Types  
%**********²*****************************************************

\begin{frame}[fragile]
  \frametitle{Types}
  \pause
  \begin{itemize}[<+->]
    \item There is a graph type in Julia reflecting the hierarchy of types
    \begin{lstlisting}
      None <: Int64 <: Number <: Real <: Any
    \end{lstlisting}
      \item support both abstract and concrete types
      \item user can annotate the code with operator \texttt{::} "is an instance of"
      \item \julia supports 
    \begin{itemize}
      \item composite types
      \item union types  
      \item tuple types  
      \item parametric types
      \item singleton types
      \item type aliases
    \end{itemize}
  \end{itemize}
\end{frame}

%****************************************************************
% Multiple dispatch 
%**********²*****************************************************

\begin{frame}[fragile]
  \frametitle{Multiple dispatch}
  \begin{itemize}
    \pause
    \item main idea : define piecewisely methods or functions depending on their arguments types
    \pause
    \item let us define \texttt{f}
    \begin{lstlisting}
      f(x::Float64,y::Float64) = 2x + y
      f(x::Int,y::Int) = 2x + y
      f(2.,3.) # returns 7.0
      f(2,3) # returns 7.0
      f(2, 3.) # throw an ERROR: no method f(Int64,Float64)
    \end{lstlisting}
    \pause
    but if we define \texttt{g}
    \begin{lstlisting}
      g(x::Number, y::Number) = 2x + y
      g(2.0, 3) # now returns 7.0
    \end{lstlisting}
    \pause
    \item no automatic or magic conversions : for operators 
    arguments are promoted to a common type (user-definable) and
    use the specific implementation
    \pause
    \item supports parametric methods
    \begin{lstlisting}
      myappend{T}(v::Vector{T}, x::T) = [v..., x] 
    \end{lstlisting}
  \end{itemize}
\end{frame}

%****************************************************************
% Performance 
%**********²*****************************************************

\begin{frame}[fragile]
  \frametitle{Performance}
  \begin{itemize}[<+->]
    \item to prove that \julia is fast language, we did some tests
    \item benchmarks sources taken from the Julia site and modified
    \item all times are in milliseconds
    \item on a Z800 (8 threads, 24G of memory), 
  \end{itemize}
  \pause
  \begin{figure}[H]
    \begin{tabular}{|c|c|c|c|c|c|c|}
      \hline
      appli & fib & mandel & quicksort & pisum & randstat & randmul \\
      \hline
      \matlab & 191 & 22 & 28 & 57 & 97 & 69 \\
      Octave & 924 & 310 & 1138 & 21159 & 484 & 109 \\
      Python &  4  & 7  & 14 & 1107 & 253 & 101 \\
      Pypy   & 8  &  3(faux) & 13 & 44 & xxx &xxx \\ 
      Julia  & 0.09 & 0.28 & 0.57 & 45 & 34 & 49 \\
      Fortran & 0.08 & $\leqslant 10^{-6}$ & 0.62 & 44 & 16 & 275(16) \\
      \hline
    \end{tabular}
  \end{figure}
\end{frame}

%**************************************************************
%  Conclusion     
%**************************************************************
\begin{frame}[fragile]{Conclusion}
  \begin{itemize}[<+->] 
    \item pros : 
    \begin{itemize}
      \item  a true language, with lots of powerful features,  
      \item \julia is fast! 
    \end{itemize}
    \item cons : 
    \begin{itemize}
      \item poor graphics support (only 2D with additional package),
      \item no support for shared-memory parallelism 
      \item small community 
    \end{itemize}
    \item some non presented points : 
    \begin{itemize}
      \item meta-programming aspects : macros 
      \item reflection 
      \item packaging system based on Git
    \end{itemize} 
    \item more info at \texttt{http://julialang.org/}
  \end{itemize} 
\end{frame}

\end{document}
